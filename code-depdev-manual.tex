\documentclass[10pt,a4paper]{article}
\usepackage[latin1]{inputenc}
\usepackage{amsmath}
\usepackage{amsfonts}
\usepackage{amssymb}
\usepackage{graphicx}
\usepackage{enumitem}
\author{Fran\c{c}ois Coppens}
\title{BCNTLS $^4$HeDFT Codebase | Quickstart Guide }
\begin{document}
	\maketitle
	\emph{An Internet connection and the `git'-program is required on the machine where the Git commands are going to be executed ! This should already be installed in most cases.}
	\section{Obtaining the code}
	To obtain the code only one command in a terminal window is required. Change to the directory where you want the top level directory of the code to reside and execute
	\begin{verbatim}
	$ git clone <GitHub URL>
	\end{verbatim}
	where \verb|<GitHub URL>| needs to be substituted by on of
	\begin{description}[align=right,labelwidth=0.35cm]
		\item[1.] \verb|https://github.com/bcntls2016/4hedft.git|
		\item[2.] \verb|https://github.com/bcntls2016/4hedft-vortex.git|
		\item[3.] \verb|https://github.com/bcntls2016/4hetddft-isotropic.git|
		\item[4.] \verb|https://github.com/bcntls2016/4hetddft-anisotropic.git|
	\end{description}

	\noindent These repositories contain, respectively,

	\begin{description}[align=right,labelwidth=0.35cm]
	\item [1.] Static $^4$Helium DFT
	\item[2.] Static $^4$Helium DFT with vortex support
	\item[3.] Time Dependent $^4$Helium DFT for impurities in an isotropic electronic state (e.g. s-states)
	\item[4.] Time Dependent $^4$Helium DFT for impurities in an anisotropic electronic state (e.g. p-states)
	\end{description}

	\section{Compiling the code}
	After git has downloaded the repository from the Internet, chance to the repository directory which should be named after the last part of the URL without the \verb|.git| part. Then execute
	\begin{verbatim}
	$ git checkout -b work
	\end{verbatim}
	This will create a new git `branch' called work where we can work in while not contaminating the `master' branch.

	You need to create a makefile to tailor it to your machines architecture. You can use one of the templates named  \verb|makefile.<machine>| if you want and save it as \verb|makefile|. After you have created your own makefile and saved it as \verb|makefile| execute
	\begin{verbatim}
	$ make
	\end{verbatim}
	After there are no errors one of \{\verb|BCN4HeDFT, scalarimp_absor_new,| \\ \verb|vectorimp_absor|\} executables should be ready to use, unless you changed the name in your makefile.

\emph{Make sure that the makefile's file name is} \verb+makefile+ \emph{otherwise your own customizations could be overwritten after an update !}

\end{document}
